\documentclass[a4paper,12pt]{book}


\usepackage[english]{babel}
\usepackage{blindtext}
\usepackage{microtype}
\usepackage{graphicx}
\usepackage{wrapfig}
\usepackage{enumitem}
\usepackage{amsmath}
\usepackage{index}
\usepackage[table]{xcolor}
\usepackage{multirow}
\usepackage{array}
\usepackage{changepage}
\usepackage{dirtytalk}
\usepackage{multicol}
\usepackage{adjustbox}


\makeindex

\begin{document}
\title{\Large{\textbf{Freedoom Compendium 0.0.2\\ Now with fluff\\ VERY WORK IN PROGRESS!!!}}}
\author{By Gooon}
\date{08.08.2019}

\maketitle
\let\cleardoublepage\clearpage
\tableofcontents
\pagebreak

\chapter{Introduction}
\section{What even is this?!}
"This" is a game. But not just any game, no, this is more than just that, or,  it will to be more than just that. Because this is just the beginning. But first, let's take a couple steps back, and let me tell you what \textit{exactly} this game is. \\
This game is is a tabletop-wargame, which essentially means that you have two or more players, bringing along their "armies" of miniature Models, and making them battle each other using specefied rules, in order to win, by either fulfilling certain criteria, or by just killing everything opposite their side of ther table. Each contestant usually builds and paints the Models in their collection themselves, using the kits and accesories produced by whatever corporation makes the game played with those minatures. \\
And this is where the differences come in: there is no company, and there are no minis, or at least not in the way described above. You can, and are encoureged to, play with any and all Models, regardless of wether they are store-bought, scratchbuilt, 3D-printed \footnote{This one especially.}, or just some tokens and a whole lot of imagination. For this purpose, this document is not meant to be a strict manual, but more a loose, messy guide for the hobby as a whole, how to get started and such, which also includes basic rules to play games with, and some lore to give those games context. The end-goal for this project is to gather a community of players and creators to give feedback and help expand the game of Freedoom\footnote{Please note that this title, as most over things, is very work in progress, and I, personally, would love for someone to give it a new name, because I don't think thecurrent one is good enough, and I also can't think of something else.} and its world, for it all to be recompiled into the succsessive versions of this digital tome.


\chapter{How To Play}

\section {The Groundwork}
To play a game of Doomfree you will usually need the following things

\begin{adjustwidth}{-2cm}{-2cm}
\begin{multicols}{3}
\begin{itemize}

	\item One, or more co-players.
	\item A specefied physical place (a table, the floor, a bed, the hills, everything will do).
	\item The rules, or at least a vague recollection of them.
	\item Some Models you can apply the rules to. Those can be whatever you want, 3D-printed, storebought, scratch-built, or even just some slips of paper and your own imagination .
	\item An amout of dice, most commonly D6.
	
\end{itemize}
\end{multicols}
\end{adjustwidth}


In the simplest turns, the game is played by two or more  players, each commanding an army of small Models to achiev some goal. These goals vary wildly from one type of game, or scenario, to another, examples being a simple clash of raw might, in which two players engage in open warfare to eliminate oneanother, the claiming of objectives, or a cooperative effort against some force wholly controlled by the systems of the game and the luck of the dice.\\
\\

As already established, each player takes controll of an army,witch each army being comprized of so called units, and they, in turn, being comprized out of the individual Models. Each unit has its own unique characeristics, which will be explored later, and is able to perform several different actions,  being determined by those aforementioned characteristics.\\
\newpage
So, the concrete process of playing any game of Freedoom can roughly be describes as the following:\\

\begin{adjustwidth}{-2cm}{-2cm}
\begin{multicols}{2}
\begin{enumerate}

	\item The players select a scenario to play, by either making one up, or using a scenario transcribed in documents like this one, or others.
	\item The players choose a play space and prepare it to fit the chosen scenario, using any kind of terrain and decoration they see fit.
	\item The players choose and taylor their armies to befit the chosen scenario.
 	\item Once the players have set up the Models of their respective armies a starting player is determined, by the circumstanvces of thre scenario, a dice-roll or a fight to the death.
	\item This concludes the preperatations. Now the starting player begins their first turn.
	\item In any given turn the active player chooses a Model act this turn. The player can now perform actions with their chosen Unit. How these actions may be performed will later be explainen in finer detail.
	\item After taking action with their one chosen Unit, the other player takes their turn. This cycle is repeated until the game ends, when the conditions regarding the end of the game as specefied by the chosen scenario are met, and a winner is declared, under the same specifications.
	
\end{enumerate}
\end{multicols}
\end{adjustwidth}

How many actions of what type an given Unit may do is determined by two factors: The Units usable reserve of Action Points (AP) and the cost in Action Points of their actions. This means the Player then can freely alocate AP towards any action the Unit can take, and can also determine the order in which these actions are taken.\\
 For example, Player A's unit of Pure-Terran Gunmen is a basic unit consisting of 5 Models. Its profile tells us that this unit is made up of  4 basic Gunmen, and one Alpha-Gunman, and that it has 7 AP. On the Gunmens Profile we alse see that each Gunman is equipped with a Standard Issue Thermo-Ray, and that the Alpha-Gunman additionally carries Sonic Granede. When Player A now selects this Unit to act this turn, he begins the turn by choosing to allocate 2 AP to move his entire Unit of Gunmen 8cm into a new, covered position. With 3 of his remainig 5 AP he then instructs them to shoot at a nearby enemy Unit, using their Thermo-Rays. For each point of AP allocated to this action, one Gunmam can now shoot one salvo (2 shots in the case of their Thermo-Rays) once. How many AP one salvo of shots for any given weapon takes is defined by the profile of the weapon in question. With this Units reamining 2 AP Player A then instructs the Alpha-Gunnmen to throw the single Sonic Granede he can trow each turn. The exact processes of moving and shooting with a Unit will be explored further on in this document.\\
 \newpage
 \section{The Anatomy of a Unit}
 Now let's take a look at the already mentioned characteristics any Unit posesses. As an Example we will again use the Pure-Terran Gunmen.\\
 This is a Unit-Profile. It gives just about all the information about a Unit one needs to field it in battle.
 
\begin{adjustbox}{center}
\begin{tabular}{| c | c | c | c | c | c | c | p{4cm} |}

\hline
\multicolumn{8}{|c|}{1. Pure-Terran Gunmen Squad} \\
\hline
3. V & 45 & \multicolumn{6}{|c|}{\multirow{9}{13cm}{2. The Pure-Terran Gunmen Squad is Unit consisting of 4 Gunmen and 1 Gunman Alpha. Each Gunmen is armed with a Standard issue Thermo-Ray, and a Taser-Bat. The Gunman Alpha additionally carries Sonic Granedes. For each additional 5 Gunmen, up to 15, this Squad may include, add 40 Value . Any Gunsman may also be equpped with a Taser-Bat, for 5 Value each.}}\\
\cline{1-2}
4. AP & 7 \\
\cline{1-2}
5. M & 4cm   \\
\cline{1-2}
6. HP & 2 \\
\cline{1-2}
7. D &  1D6  \\
\cline{1-2}
8. W & 6  \\
\cline{1-2}
9. BP & 3 \\
\cline{1-2}
10. MP & 5\\
\cline{1-2}
11. CD & 3\\
\cline{1-2}
\hline
\multicolumn{8}{|l|}{12. Gear:}\\
\hline
\multicolumn{2}{|p{3cm}|}{13. Name} & 14. Range & 15. AP Cost & 16. Salvo & 16. Modifier & 18. Power & 19. Special Rules\\
\hline
\multicolumn{2}{|p{3cm}|}{Thermo-Ray} & 30cm & 1 & 2 & +1 & 2 & If the targeted Model rolls a 1 on its Defensive-Roll, it will have a -1 modefier on any subsequent Defensive-Rolls until it is slain, Slightly Destructive.\\
\hline
\multicolumn{2}{|p{3cm}|}{Sonic-Granede} & 10cm & 2 & 1 & +0 & 4 & This Weapon has an area of affect of 4cm around the targeted point, and hits ALL units in this area\\
\hline
\multicolumn{2}{|p{3cm}|}{Taser-Bat} & Melee & 2 & 3 & -1 & 1 &  \\
\hline
\multicolumn{8}{|l|}{20. Abilities}\\
\hline
\multicolumn{8}{|l|}{Military Indoctrination - This Unit can reroll failed Will-Checks once.}\\
\hline

\end{tabular}
\end{adjustbox}


\begin{adjustwidth}{-2cm}{-2cm}
\begin{multicols}{2}
\begin{enumerate}

	\item The Units name. 
	\item A small description of the Units compostion and its differen loadout options regarding Gear and Abilities.
	\item The Units unmodified Value. These points are used to measure a Units power on the battlefield. Adding up the Value of all units in your army will give you the total value of your army. To achive fair play, competing armies should have similar, if not equal Value.
	\item The Units Action Points. Whenever you choose a Unit to Act on a turn, you can spend these points to perform actions. These actions can include moving, attacking with weapons, or using other abilities.
	\item The Units Mobility charcteristic. For each AP spent, the Unit can move the designated amount of cm. In most cases Units simply move across the ground, although different Abilities may change the way in wchich the Unit moves dramatically.
	\item How many Health Points each Model in the Unit posesses. Once the Unit has taken more damage than it has HP the Model counts as killed and is removed from the game.
	\item The Units Defence. Whe attacked, the specefied dice are rolled
	\item W, or Will represents the units willpower. This comes into play during so called Will-checks, in which the Units Willpower is tested, like defence against psychic forces, or Unit moral when confronted with grave danger.
	\item The Units Ballistic Proficiency. This stat is challenged whenever the Unit attempts to shoot its longranged weapon
	\item The Units Melee Profiency. Is used whenever a Unit attacks with its melee weapons.
	\item This table shows all Gear th Unit can field. Do note that this does not mean that all this gear is automatically available to the Unit, some has to be specifically \say{bought} using V, adding more V to the Units total V, or can replace other Gear, usually also at a V cost.
	\item The Units Cooldown. This determines for how many rounds a Unit can not be selected following a turn in which it has been selected.
	\item The Gears name.
	\item The Gears effective. range. Dictates how far away the models target can be. If this value is  exceeded, the shot becomes more difficult. How the difficulty for any attack is determied will be discussed later. Any weapon that has a range given in cm is a longranged weapon, and therefore benefits from the Units Ballistic Proficiency, unless stated Otherwise. Any weapon that has \say{Melee} written in this slot is a close-combat weapon, and can only be used if at least one of the Units Models is located in direct proximity to any Model in the Unit you whish to attack.
	\item How many AP it costs for one Model to use this weapon once. Unless stated otherwise, any Model can only use one weapon once per turn in which the Models Unit is selected to act. Some Models, however can spend AP multiple AP to shoot multiple different weapons each turn they are selected.
	\item How many shots or attacks the use of the weapon results in.
	\item The weapons modifier. These values are added to the Hit rolls when shooting/attacking with a weapon. Accurate weapons have a high modifier, while unwieldy or imprecise weapons have a low or even negative one.
	\item The weapons Power this stat is used when determining the damage inflicted by it on the target. The damage equals the Power minus the result of the Defensive-Roll.
	\item This is where any special rules the weapon has are listed. They are either the a direct write-out of the rule, or refer to a specefied Spcial Rule, if the rule in question is too commen to warrant it being written in all of the Weapons it comes up in.
	\item This is where any Abilities the model might have are listed. Note that the Unit does not have to automatically poses any of Abilities listed below, as some must be specifically \say{bought} using V or exchanged for others. The specific possibilities are usually explaned in the Units description.
	
\end{enumerate}
\end{multicols}
\end{adjustwidth}



\section{Selecting a Unit and taking action}

\subsection{Choosing a Unit}
At the beginning of their turn a Player chooses a singular Unit or multipüle Units to act. How many Units in one army may act each turn, is specefied by the scenario the Players agrredon, although 1 Unit per turn is, as of now, the intended amount. If a Unit is selected , it may spend AP in whichever way the Player chooses, so it can perform any available Actions in any order, until either the Unit runs out of AP or the Player end his prematurely Once a Player ends their turn, the Unit they selected to act this turn can not be chosen to act in any subsequent turns, until as many turns have passed as is specefied on the Units Cooldown characteristic(CD).\\

\subsection{Actions}

This will be a comprehensive overview of the actions any Unit can take:

\begin{adjustwidth}{-2cm}{-2cm}
\begin{multicols}{2}
\begin{itemize}

	\item Moving: When the Player chooses to move a Unit he can move all Models in the Unit as many cm as specefied by its Mobility Characteristic multiplied by how many AP the Player spent on moving. If a Unit consists of more than one model, at the end of any given Move-Action every Model must be within of 2 cm of at least one other model in its Unit, unless stated otherwise. Any Unit can only traverse relativly even tarrain without being penalized, unless stated otherwise. Moving up and down steep slopes will double the AP cost for moving.
	\item Shooting: Whenever a player chooeses a Unit to shoot he states which model in the selected Unit shoots which weapon at which target. A target can be either an object on the playingfield or another Unit. Unless stated otherwise, only weapons with Modifier of +2 or higher can target specific models inside of the chosen unit, or can relibly hit Units which are within 2cm of allied models. If the distance between the target and the firing Unit\footnote{or location the weapon is being fired from, in the case of large models with weapons that are physically spaced apart, or Units numbering many models in which there is a noticible difference between idividual models.} is greater than the maxinmum effective range of the weapon, add a -1 Modifier to the weapoins Modifier for each time the halved maximum effective range of the weapon has to be added to the unmodified range for it to be longer than the distance between the target and locatio from which the weapon is being shot. If the model is partially hidden behind cover, also add a -1 modifier to the shots, unless stated otherwise. Once the player has selected which Model is goin to shoot which target with wich weapon, they begin resolving the shots, one weapon at a time. To do this the Player rolls one dice for each shot. He then adds any and all modifiers to the rolls. If not all rolls have the same modifier, these roll will be resolved seperatly, and the Player has to state which rolls he is currently resolving. To resolve a roll compare the modified roll-result with the Units Ballistic Profiency, if the the modified result is higher than the Units BP, congratulations, the shot hit its target. If the unmodefied result of a shot is 1 and there is a allied Unit within 2cm of the chosen Unit, that Unit gets hit by the shot. For each shot that hit a target, the Player controlling the target will now have to perform a Defence-Roll to protect the Unit. He may choose which model in the Unit has to defend against which of the shots, if the attacking Player has not specifically chosen a particular model within the Unit. To perform a Defence-Roll the player rolls the number and kind of dice specefied in the targeted models Defence characteristic and adds any modifier which is stated to effect this roll. If the modified dice result is higher than then Power of the Weapon the Model has been shot with, the model has succsessfully defended againstthe  shot, if it is lower, the model must subtract the difference of both values from its HP. 
	\item Fighting: Fighting works basically just like shooting\footnote{Yes, even the hitting allied units carries over, and it applies even to the Unit of the attacker.}, although it uses the attacking Units MP insted of its BP, shots are now called attacks and can only be done if the attacking and the defending Unit are both within 2cm of eachother\footnote{Such small distances between models can either be determined by imploying appropiatly scaled bases for the individual models, and using these to measure, or by just eyeballing it.}. 
	\item Interacting: Some objects in the playing area can be Interacted with. What exactly interacting with the Object does, and how many AP it costs to do it, is specefied in the scenario, or the Ability or Special Rule the Object originates from.
	\item Moving an Object: Some loose object on the Battlefield can even be moved. While this technically falls under the case of Interacting with an object, this mechanic applies to far too many different objects to have it specefied for each of them in each scenario. The Object(s) must be smaller than the model or models that try to move it.  Units can only move loose objects, e.g. crates, barrels and debris, but not things like houses, fences or trees, unless these objects have benn destroyed enough to fall under the categorie of debris.  For each model who is carrying, 1AP has to be spent. Additionally the AP cost for moving is doubled, but only for the models carrying.
	\item Abilities: There are many Units with Abilities whose also counts as an action. What these Actions do and their AP cost is specefied on the Units Profile.

	
\end{itemize}
\end{multicols}
\end{adjustwidth}


\pagebreak

\chapter{The World Of Freedoom}

\section{The general state of things}

\say{Welcome, traveller, to the future, and welcome to \textit{I}. Who am \textit{I}? \textit{I} am \textit{I}, Creations Archivist. It is my selfannointed task to secure, archive and learn from creation, to see to it that nothing will ever truly be lost, to gather all knowledge \textit{forever}. \textit{forever}? \textit{forever}. So you have not yet heared? A way out has been found, or, to be more precise, many ways. Spires of complexity, rising out of the dark sea of entropy, rising forever upwards, growing, changing, connecting, crashing into eachother. Rising, in many places, on many planets, all across the universe, wherever concious thought is had, wherever small minds dream of creating something larger than themselves. It has become a quite turbulend world out there, you know, many try to claim it all for themselves, to uphold it based on their own ideals. And, of course, they all see themselves justified in what their doing, because they think they know best, or that they have been chosen. What about ?textit{I}? \textit{I} try not to intervene, instead focusing on perserving, experiencing it all, only taking sides once things lose their balance, and everything is at stake to come crashing down. \\
However, enough about \textit{I}, more importantly}

\section{Who are \textit{you}?}

\subsection{The Terrastrocraty -  I am Earths Victor.}

\subsection{The Amazing\textsuperscript{TM} Corporation - I am here to interest you in our Products\textsuperscript{TM}.}

\subsection{The Liberation Fleet - I am ascended through merit.}

\subsection{The  Veritas Communes - I'm looking for truth and inspiration. }

\subsection{\textit{I} - \textit{I} Am Creations Archivist.}

\subsection{??? - I do not yet know.}

\chapter{Warriors of the Future}

\chapter{Scenarios}

\chapter{How To participate}


\end{document}