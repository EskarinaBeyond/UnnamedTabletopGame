\documentclass[a4paper,12pt]{report}


\usepackage[english]{babel}
\usepackage{blindtext}
\usepackage{microtype}
\usepackage{graphicx}
\usepackage{wrapfig}
\usepackage{enumitem}
\usepackage{amsmath}
\usepackage{index}
\usepackage[table]{xcolor}
\usepackage{multirow}
\usepackage{array}
\usepackage{changepage}
\usepackage{dirtytalk}
\usepackage{multicol}
\usepackage{adjustbox}


\makeindex

\begin{document}
\title{\Large{\textbf{Untitled Tabletop Game Core-Compendium 0.0.5\\ tl;dr: The god-space gave every living thing in existence  OCD.\\ VERY WORK IN PROGRESS!!!}}}
\author{By Gooon}
\date{08.08.2019}

\maketitle
\let\cleardoublepage\clearpage
\tableofcontents
\pagebreak

\chapter{Introduction}
\section{What even is this?!}
"This" is a game. But not just any game, no, this is more than just that, or,  it will to be more than just that. Because this is just the beginning. But first, let's take a couple steps back, and let me tell you what \textit{exactly} this game is. \\
This game is is a tabletop-wargame, which essentially means that you have two or more players, bringing along their "armies" of miniature Models, and making them battle each other using specified rules, in order to win, by either fulfilling certain criteria, or by just killing everything opposite their side of their table. Each contestant usually builds and paints the Models in their collection themselves, using the kits and accessories produced by whatever corporation makes the game played with those miniatures. \\
And this is where the differences come in: there is no company, and there are no minis, or at least not in the way described above. You can, and are encouraged to, play with any and all Models, regardless of whether they are store-bought, scratch-built, 3D-printed \footnote{This one especially.}, or just some tokens and a whole lot of imagination. For this purpose, this document is not meant to be a strict manual, but more a loose, messy guide for the hobby as a whole, how to get started and such, which also includes basic rules to play games with, and some lore to give those games context. The end-goal for this project is to gather a community of players and creators to give feedback and help expand the game of Untitled Tabletop Game\footnote{Please note that this title, as most over things, is very work in progress, and I, personally, would love for someone to give it a new name, because I don't think the current one is good enough, and I also can't think of something else.} and its world, for it all to be recompiled into the successive versions of this digital tome.


\chapter{How To Play}

\section {The Groundwork}
To play a game of Untitled Tabletop Game you will usually need the following things

\begin{adjustwidth}{-2cm}{-2cm}
\begin{multicols}{3}
\begin{itemize}

	\item One, or more co-players.
	\item A specified physical place (a table, the floor, a bed, the hills, everything will do).
	\item The rules, or at least a vague recollection of them.
	\item Some Models you can apply the rules to. Those can be whatever you want, 3D-printed, store-bought, scratch-built, or even just some slips of paper and your own imagination .
	\item An amount of dice, most commonly D6.
	
\end{itemize}
\end{multicols}
\end{adjustwidth}


In the simplest turns, the game is played by two or more  players, each commanding an army of small Models to achieve some goal. These goals vary wildly from one type of game, or scenario, to another, examples being a simple clash of raw might, in which two players engage in open warfare to eliminate one another, the claiming of objectives, or a cooperative effort against some force wholly controlled by the systems of the game and the luck of the dice.\\
\\

As already established, each player takes control of an army,witch each army being comprised of so called units, and they, in turn, being comprised out of the individual Models. Each unit has its own unique characteristics, which will be explored later, and is able to perform several different actions,  being determined by those aforementioned characteristics.\\
\newpage
So, the concrete process of playing any game of Untitled Tabletop Game can roughly be describes as the following:\\

\begin{adjustwidth}{-2cm}{-2cm}
\begin{multicols}{2}
\begin{enumerate}

	\item The players select a scenario to play, by either making one up, or using a scenario transcribed in documents like this one, or others.
	\item The players choose a play space and prepare it to fit the chosen scenario, using any kind of terrain and decoration they see fit.
	\item The players choose and tailor their armies to befit the chosen scenario.
 	\item Once the players have set up the Models of their respective armies a starting player is determined, by the circumstances of the scenario, a dice-roll or a fight to the death.
	\item This concludes the preparations. Now the starting player begins their first turn.
	\item In any given turn the active player chooses a Model act this turn. The player can now perform actions with their chosen Unit. How these actions may be performed will later be explaining in finer detail.
	\item After taking action with their one chosen Unit, the other player takes their turn. This cycle is repeated until the game ends, when the conditions regarding the end of the game as specified by the chosen scenario are met, and a winner is declared, under the same specifications.
	
\end{enumerate}
\end{multicols}
\end{adjustwidth}

How many actions of what type an given Unit may do is determined by two factors: The Units usable reserve of Action Points (AP) and the cost in Action Points of their actions. This means the Player then can freely allocate AP towards any action the Unit can take, and can also determine the order in which these actions are taken.\\
 For example, Player A's unit of Terran Gunmen is a basic unit consisting of 5 Models. Its profile tells us that this unit is made up of  4 basic Gunmen, and one Alpha-Gunman, and that it has 7 AP. On the Gunmens Profile we also see that each Gunman is equipped with a Standard Issue Thermo-Ray, and that the Alpha-Gunman additionally carries Sonic Grenade. When Player A now selects this Unit to act this turn, he begins the turn by choosing to allocate 2 AP to move his entire Unit of Gunmen 8cm into a new, covered position. With 3 of his remaining 5 AP he then instructs them to shoot at a nearby enemy Unit, using their Thermo-Rays. For each point of AP allocated to this action, one Gunmam can now shoot one salvo (2 shots in the case of their Thermo-Rays) once. How many AP one salvo of shots for any given weapon takes is defined by the profile of the weapon in question. With this Units remaining 2 AP Player A then instructs the Alpha-Gunnmen to throw the single Sonic Granede he can throw each turn. The exact processes of moving and shooting with a Unit will be explored further on in this document.\\
 \newpage
 \section{The Anatomy of a Unit}
 Now let's take a look at the already mentioned characteristics any Unit possesses. As an Example we will again use the Terran Gunmen.\\
 This is a Unit-Profile. It gives just about all the information about a Unit one needs to field it in battle.
 
\begin{table}[!ht]
\begin{adjustwidth}{-3cm}{0cm}
\begin{tabular}{| c | c | c | c | c | c | c | p{4cm} |}

\hline
\multicolumn{8}{|c|}{1. Terran Gunmen Squad} \\
\hline
3. V & 45 & \multicolumn{6}{|c|}{\multirow{9}{13cm}{2. The Terran Gunmen Squad is Unit consisting of 4 Gunmen and 1 Gunman Alpha. Each Gunmen is armed with a Standard issue Thermo-Ray, and a Taser-Bat. The Gunman Alpha additionally carries Sonic Grenades. For each additional 5 Gunmen, up to 15, this Squad may include, add 40 Value . Any Gunsman may also be equipped with a Taser-Bat, for 5 Value each.}}\\
\cline{1-2}
4. AP & 7 \\
\cline{1-2}
5. M & 4cm   \\
\cline{1-2}
6. HP & 2 \\
\cline{1-2}
7. D &  1D6  \\
\cline{1-2}
8. W & 6  \\
\cline{1-2}
9. BP & 3 \\
\cline{1-2}
10. MP & 5\\
\cline{1-2}
11. CD & 3\\
\cline{1-2}
\hline
\multicolumn{8}{|l|}{12. Gear:}\\
\hline
\multicolumn{2}{|p{3cm}|}{13. Name} & 14. Range & 15. AP Cost & 16. Salvo & 16. Modifier & 18. Power & 19. Special Rules\\
\hline
\multicolumn{2}{|p{3cm}|}{Thermo-Ray} & 30cm & 1 & 2 & +1 & 2 & If the targeted Model rolls a 1 on its Defensive-Roll, it will have a -1 modifier on any subsequent Defensive-Rolls until it is slain, Slightly Destructive.\\
\hline
\multicolumn{2}{|p{3cm}|}{Sonic-Grenade} & 10cm & 2 & 1 & +0 & 4 & This Weapon has an area of affect of 4cm around the targeted point, and hits ALL units in this area\\
\hline
\multicolumn{2}{|p{3cm}|}{Taser-Bat} & Melee & 2 & 3 & -1 & 1 &  \\
\hline
\multicolumn{8}{|l|}{20. Abilities}\\
\hline
\multicolumn{8}{|l|}{Military Indoctrination - This Unit can reroll failed Will-Checks once.}\\
\hline

\end{tabular}
\end{adjustwidth}
\end{table}

\pagebreak

\begin{adjustwidth}{-2cm}{-2cm}
\begin{multicols}{2}
\begin{enumerate}

	\item The Units name. 
	\item A small description of the Units composition and its different load-out options regarding Gear and Abilities.
	\item The Units unmodified Value. These points are used to measure a Units power on the battlefield. Adding up the Value of all units in your army will give you the total value of your army. To achieve fair play, competing armies should have similar, if not equal Value.
	\item The Units Action Points. Whenever you choose a Unit to Act on a turn, you can spend these points to perform actions. These actions can include moving, attacking with weapons, or using other abilities.
	\item The Units Mobility characteristic. For each AP spent, the Unit can move the designated amount of cm. In most cases Units simply move across the ground, although different Abilities may change the way in wchich the Unit moves dramatically.
	\item How many Health Points each Model in the Unit possesses. Once the Unit has taken more damage than it has HP the Model counts as killed and is removed from the game.
	\item The Units Defence. When attacked, the specified dice are rolled.
	\item W, or Will represents the units willpower. This comes into play during so called Will-checks, in which the Units Willpower is tested, like defence against psychic forces, or Unit moral when confronted with grave danger.
	\item The Units Ballistic Proficiency. This stat is challenged whenever the Unit attempts to shoot its long-ranged weapon
	\item The Units Melee Proficiency. Is used whenever a Unit attacks with its melee weapons.
	\item This table shows all Gear the Unit can field. Do note that this does not mean that all this gear is automatically available to the Unit, some has to be specifically \say{bought} using V, adding more V to the Units total V, or can replace other Gear, usually also at a V cost.
	\item The Units Cooldown. This determines for how many rounds a Unit can not be selected following a turn in which it has been selected.
	\item The Gears name.
	\item The Gears effective. range. Dictates how far away the models target can be. If this value is  exceeded, the shot becomes more difficult. How the difficulty for any attack is determined will be discussed later. Any weapon that has a range given in cm is a long-ranged weapon, and therefore benefits from the Units Ballistic Proficiency, unless stated Otherwise. Any weapon that has \say{Melee} written in this slot is a close-combat weapon, and can only be used if at least one of the Units Models is located in direct proximity to any Model in the Unit you wish to attack.
	\item How many AP it costs for one Model to use this weapon once. Unless stated otherwise, any Model can only use one weapon once per turn in which the Models Unit is selected to act. Some Models, however can spend AP multiple AP to shoot multiple different weapons each turn they are selected.
	\item How many shots or attacks the use of the weapon results in.
	\item The weapons modifier. These values are added to the Hit rolls when shooting/attacking with a weapon. Accurate weapons have a high modifier, while unwieldy or imprecise weapons have a low or even negative one.
	\item The weapons Power this stat is used when determining the damage inflicted by it on the target. The damage equals the Power minus the result of the Defensive-Roll.
	\item This is where any special rules the weapon has are listed. They are either the a direct write-out of the rule, or refer to a specified Spcial Rule, if the rule in question is too common to warrant it being written in all of the Weapons it comes up in.
	\item This is where any Abilities the model might have are listed. Note that the Unit does not have to automatically poses any of Abilities listed below, as some must be specifically \say{bought} using V or exchanged for others. The specific possibilities are usually explained in the Units description.
	
\end{enumerate}
\end{multicols}
\end{adjustwidth}



\section{Selecting a Unit and taking action}

\subsection{Choosing a Unit}
At the beginning of their turn a Player chooses a singular Unit or multiple Units to act. How many Units in one army may act each turn, is specified by the scenario the Players agreed on, although 1 Unit per turn is, as of now, the intended amount. If a Unit is selected , it may spend AP in whichever way the Player chooses, so it can perform any available Actions in any order, until either the Unit runs out of AP or the Player end his prematurely Once a Player ends their turn, the Unit they selected to act this turn can not be chosen to act in any subsequent turns, until as many turns have passed as is specified on the Units Cooldown characteristic(CD).\\

\subsection{Actions}

This will be a comprehensive overview of the actions any Unit can take:

\begin{adjustwidth}{-2cm}{-2cm}
\begin{multicols}{2}
\begin{itemize}

	\item Moving: When the Player chooses to move a Unit he can move all Models in the Unit as many cm as specified by its Mobility Characteristic multiplied by how many AP the Player spent on moving. If a Unit consists of more than one model, at the end of any given Move-Action every Model must be within of 2 cm of at least one other model in its Unit, unless stated otherwise. Any Unit can only traverse relatively even terrain without being penalized, unless stated otherwise. Moving up and down steep slopes will double the AP cost for moving.
	\item Shooting: Whenever a player chooses a Unit to shoot he states which model in the selected Unit shoots which weapon at which target. A target can be either an object on the playing-field or another Unit. Unless stated otherwise, only weapons with Modifier of +2 or higher can target specific models inside of the chosen unit, or can reliably hit Units which are within 2cm of allied models. If the distance between the target and the firing Unit\footnote{or location the weapon is being fired from, in the case of large models with weapons that are physically spaced apart, or Units numbering many models in which there is a noticeable difference between individual models.} is greater than the maximum effective range of the weapon, add a -1 Modifier to the weapons Modifier for each time the halved maximum effective range of the weapon has to be added to the unmodified range for it to be longer than the distance between the target and location from which the weapon is being shot. If the model is partially hidden behind cover, also add a -1 modifier to the shots, unless stated otherwise. Once the player has selected which Model is going to shoot which target with which weapon, they begin resolving the shots, one weapon at a time. To do this the Player rolls one dice for each shot. He then adds any and all modifiers to the rolls. If not all rolls have the same modifier, these roll will be resolved separately, and the Player has to state which rolls he is currently resolving. To resolve a roll compare the modified roll-result with the Units Ballistic Proficiency, if the the modified result is higher than the Units BP, congratulations, the shot hit its target. If the unmodified result of a shot is 1 and there is a allied Unit within 2cm of the chosen Unit, that Unit gets hit by the shot. For each shot that hit a target, the Player controlling the target will now have to perform a Defence-Roll to protect the Unit. He may choose which model in the Unit has to defend against which of the shots, if the attacking Player has not specifically chosen a particular model within the Unit. To perform a Defence-Roll the player rolls the number and kind of dice specified in the targeted models Defence characteristic and adds any modifier which is stated to effect this roll. If the modified dice result is higher than then Power of the Weapon the Model has been shot with, the model has successfully defended against the  shot, if it is lower, the model must subtract the difference of both values from its HP. 
	\item Fighting: Fighting works basically just like shooting\footnote{Yes, even the hitting allied units carries over, and it applies even to the Unit of the attacker.}, although it uses the attacking Units MP instead of its BP, shots are now called attacks and can only be done if the attacking and the defending Unit are both within 2cm of each-other\footnote{Such small distances between models can either be determined by employing appropriately scaled bases for the individual models, and using these to measure, or by just eyeballing it.}. 
	\item Interacting: Some objects in the playing area can be Interacted with. What exactly interacting with the Object does, and how many AP it costs to do it, is specified in the scenario, or the Ability or Special Rule the Object originates from.
	\item Moving an Object: Some loose object on the Battlefield can even be moved. While this technically falls under the case of Interacting with an object, this mechanic applies to far too many different objects to have it specefied for each of them in each scenario. The Object(s) must be smaller than the model or models that try to move it.  Units can only move loose objects, e.g. crates, barrels and debris, but not things like houses, fences or trees, unless these objects have been destroyed enough to fall under the category of debris.  For each model who is carrying, 1AP has to be spent. Additionally the AP cost for moving is doubled, but only for the models carrying.
	\item Abilities: There are many Units with Abilities whose also counts as an action. What these Actions do and their AP cost is specefied on the Units Profile.

	
\end{itemize}
\end{multicols}
\end{adjustwidth}


\pagebreak

\chapter{The World Of Untitled Tabletop Game}

\section{The general state of things}

\say{Welcome, traveller, to the future, and welcome to \textit{I}. Who am \textit{I}? \textit{I} am \textit{I}, Creations Archivist. It is my self-annointed task to secure, archive and learn from creation, to see to it that nothing will ever truly be lost, to gather all knowledge \textit{forever}. \textit{forever}? \textit{forever}. So you have not yet heard? A way out has been found, or, to be more precise, many ways. Spires of complexity, rising out of the dark sea of entropy, rising forever upwards, growing, changing, connecting, crashing into each-other. Rising, in many places, on many planets, all across the universe, wherever conscious thought is had, wherever small minds dream of creating something larger than themselves. It has become a quite turbulent world out there, you know, many try to claim it all for themselves, to uphold it based on their own ideals. And, of course, they all see themselves justified in what their doing, because they think they know best, or that they have been chosen. What about \textit{I}? \textit{I} try not to intervene, instead focusing on preserving, experiencing it all, only taking sides once things lose their balance, and everything is at stake to come crashing down.\\
\textit{But how is this possible?} A curious one, aren't you? You see, besides, or to be more precise above, the classical four dimensions this universe exists in, which is to say, three spatial ones and time, there is more. An infinite space, untethered by the rules we experience here in this side of the universe, shaped, and facilitated by the power \textit{complexity}. It connects all that is, was and will be and wherever patterns emerge, where structure is rises against entropy through random chance this space is evoked, and if this complexity reaches a high enough amount, it can even pierce onto this world and engage in its processes. These dimensions is what we call the Meta-Space. However, its function and existence is tethered to the complexities of our world, and where no complexity exists, it can not exist, and where complexity is destroyed, it is destroyed. The Meta-Space is also the answer to the riddle of sentience: wherever the mind of a beast or the circuitry of a computer reach critical complexity, they too reach out of this world and into this higher one, where its unknown processes are responsible for creating self-awareness, or the soul, as it is most commonly known. It is believed that we, as sentient beings, are a way for Meta-Space, and by extension, the universe, to perpetuate itself, to not let the universe fall down into an entropic darkness, devoid of purpose and meaning trough our innate urges to seek out and create patterns and structure. On the other hand, a curtain balance must be struck, then if the, infinite, boundless energies of the Meta-Space are siphoned into our dimensions carelessly and without measure, critical mass could be reached and the entire universe could be swallowed  by a massive black hole, destroying this universe everything in the process and leaving Meta-Space an endless, empty expanse. Some also say that an elevation from this plane into the higher ones is possibly, by creating something of unimaginable high complexity to serve as a vessel for their soul to inhabit in Meta-Space, although few claim to have achieved this state, and ever fewer have any evidence for their claims.
However, enough about the world out there, more importantly}

\pagebreak

\section{Who are \textit{you}?}




\subsection{The Terrastrocraty -  I am Earths Victor.}
Even though the legacy of the Humans of Earth can be traced back far further than most other races, this noble species made its first steps onto the greater galactic stage not a few hundred of their years ago. This group deems itself chosen by their home-planet, the greatness and glory of it they verify in all aspects of their culture, and through which they legitimize their many crusades, by the fact that they were the ones on their planet to dominate all others after millennia of active warfare, and claim victory over this holy-land, which lays at the precise centre of their growing empire. They are united under a singular rule, enforced by the royal family,which is crowned by The King and The Queen, a legendary figures, who, so they are exalt, won The Last World War, using their immense tactical prowess and innate talent of combat. They now sit upon their Throne of Terra, in the Gaian palace, which spans every last meter of the planet, doubling as the Terrastrocaties prime castle, from which they govern their vast empire, and coordinate all of its military efforts. They are each-others equals in all  measures, and are said to live forever, bringing never-ending prosperity to their empire. Even though the King and Queen have no succession to plan for, they still gave rise to numerous Princes and Princesses, who, once they have met maturity, are sent away to conquer themselves a world to rule, and with which to expand the empire. These Princes and Princesses are given relative freedom as to how exactly they rule their domains, as long as they orientate themselves after the holy example set by Earth. \\

\subsubsection*{A Society ruled from above}

The common population of the Terrastrocraty is unwaveringly devoted to their royal family, a state achieved through the widespread use of information-control and propaganda, assuring that no words of the empires losses reach their ears, while each of its victories is proclaimed trough every channel. Above the common folk, and often being served by them, are the lower nobles, humans of higher status, who are allowed to take up various academic professions, like general, scientist, architect, artist or, technician, although they additionally function as the royal families informants and advisors, their eyes and ears. Every noble is also brainwashed, as to always speak the truth in front the royals, as to not lie to them to further their own goals. This group itself is splintered and split into various sub-ranks to numerous to name and describe, filling up rank upon rank on the hierarchy.\\
\\
The very top of this society is preserved for one specific group alone: the royals. Only further divided the two generations that comprise them. They are the undisputed rulers of the empire, controlling everything from commerce with other galactic groups, to the colonization  of new planets. Every Prince and every Princess is an extension of their progenitors, and must adhere strictly to their doctrine, and to break with it, is to break with the empire itself, an action which will lead to a swift demise at the hands of the Royal Guard. Furthermore, while Princes and Princesses are allowed to take partners and even have children, none of these children are considered heirs to their throne. While this may colour the relationship between the ruling monarchs and their children as cruel, and distant, it is a widely known truism that each of the King and Queens children is loved equally by heir parents, and that to conscript them to a lifelong service for the empire is simply for the good of all, and that every reluctant Princess or Prince will eventually find to their calling to rule. Whenever a Prince or Princess dies, their spot is taken by one of the Princes or Princesses on Earthy, most commonly the oldest, unless they were to pass on that honour to their next-youngest sibling, in order to gain their glory, by conquering a new world. Once a planet has been settled in such a manner, all of its efforts are devoted to supplying the empires war-machine with either manpower or weaponry, depending on the planets usage as either a Factory- or Garden-World. The prier being comprised mainly of farmland and Hab-Blocks, while the latter is covered in vast factories, forges and military training camps, and is densely orbited by countless space-docks, which are all mostly automated as not to divert manpower from the empires narrow focus of expansion.\\

\subsubsection*{An Armada of Expansion}
To fuel the growth of its domain the Tarrastrocty implores an endless armada of warships and freighters, called the Terran Forces, to crush any stellar opposition and conquer any selected world in a planetary assault. The Terran forces are split up in various Divisions, each focusing on claiming one planet at a time. For each new siege a new Division is instantiated, and after its victory the Division is disbanded, and its ships and their crews are sent to either joint the creation of new divisions, or reinforce existing ones. Once a suitable, earth-like planet has been located, they begin their assault using the newly gathered forces to overrun the enemies orbital defences, and swiftly descend onto the planets surface where they will begin to siege important enemy structures. Even though the Terrastrocraty tries to have any gives war last as little as possible to quickly regain any losses they may have sustained using the planets resources, these sieges have been observed to last numerous decades, especially when the besieged world is home to a part of the Veritas Commune. In such cases the Terran forces simply tend to increase the number of ships being sent as reinforcements, until the enemy gives up, faced with the sheer inexhaustibility of their attackers. Each division is lead by the Princes and Princesses that supply the war with the products of their world, so either supplies and manpower, or weaponry and training. This makes it so that for each division, at least two royal mus be present. This is done to keep them from conquering new worlds, as any rogue behaviour will quickly be suppressed, as both royals are incentivized to monitor each other, and are to report on the other one to their parents. This is has the additional benefit that any traitorous Princess or Prince would have to kill their kin, if they want to take control of an army, putting an heavy emotional toll on their back, seeing as they are raised to love and protect their family unconditionally.\\
\\
Having no regard for enemy structures, as they see them as extensions of their culture, and therefore as inferior, but also not wanting to risk permanently damaging the planets environment and atmosphere, as they want to settle it as quickly as possible,the Terrarstrocraty's scientists developed the so-called Thermo-Weapons, which function by emitting ultra-high frequency electro-magnetic waves in focused beams, which will not only destroy any organic matter it hits, but also deteriorates armour and machinery, to wear down the enemy's defensive and offensive capabilities to secure a decisive victory. To feed these weapons massive demand of of energy, The Terran Forces use so-called Pattern-Engines, machines which function by creating a small opening into Meta-Space. This small dimensional rift is opened and maintained by an ornate, spherical structure, with a probe in the middle of it, which reaches into Meta-Space and siphons of some of it's energy into the weapons batteries. These Thermo-Weapons can vary in size from small limited shot pistols which have to be manually recharged, to enormous energy cannons mounted to interstellar vessels, meant to orbitally annihilate the enemies key structures. There exist also various specialized pieces of Thermo-Weaponry, like short-ranged Thermo-Torches, or the Siege-Drill, which fires a continuous, if immobile, beam of energy, to melt into enemy fortifications, and create openings for the troops.\\
\\
To win their assaults the Terrastrocraty tries to overrun their enemies defences using the sheer volume of troops at their disposal, seeing as 90\% of their population is automatically conscripted into their military forces, to serve until they die. This, however, does  not mean that their wars are fought by an onslaught of untrained cannon-fodder, far from it: all Terran boys and girls are, once they reach maturity, shipped to the next available factory-World, where they will spent the next two years undergoing rigorous training and conditioning in one of the planets massive training facilities. Even though the mortality rates of these camps lie at around 10\%, every Terran commoner sees it as their exalted duty and glorious destiny to serve in their empires ongoing expansion, even if it will cause them immense suffering or, inevitably kill them. \say{Be strong for your Empire, as it is strong for you} is a proverb often in Terran culture which most correctly displays their relationship with their service. While most of the empires soldiers are basic Gunmen or variations thereof, there exist an more elite class of warrior, the Paladins. These warriors are clad in a suit of Exo-Armour, a powerful piece of gear, designed to enhance the wearers, or pilots, in case of of the larger models, combat abilities, increasing their strength, mobility and protection. These Paladins are the answer to the large automated war-machines used by other factions, as Monarchs believe staunchly in the notion that the Aristrocraties wars should be won by its people, and not some inhuman, and therefore inferior automaton. Paladins, as already stated, come in various sizes, some not much taller than an ordinary human, while others tower over the battlefields as monstrous constructs of war. These mechanized suits of armour reserved only for the nobles of aristrocratorien society, to protect them while they attend to their duties on the battlefield, like repairing machinery, tending to wounded soldiers or more active tasks, like taking out key targets in the enemies army, or leading the charge from the front lines. While the ownership of a Paladin is not inherently tied to the users place in Terran society, practically it is almost impossible for most soldiers to to acquire more than a simple, unarmoured exoskeleton, yet alone maintain it, given their high prices and the required know-how.The most powerful of the Paladins are piloted by the Royals, and are usually represent the center-piece of any army, and are equipped with the most powerful weaponry the Factory-Worlds can provide. Each of these royal paladins is tailor-made for its one and only wearer, and often represents their own personality as much as iz represents the might of the Terrastrocraty at large. All of these war-suits are of vaguely humanoid appearance, seeing as the human form is  valued as the strongest by the empire for having been the one to conquer their home planet of earth, but some differ a fair bit from this template, sporting additional limbs, wings or other inhuman qualities. Paladins are the pride of Terran engineering, and a key part of their culture in the Aristrocraty and and, as such, are seen as symbols of might and status, even outside the battlefield. Many nobles are known to parade around in their lavishly ornamented Paladins, and often engage in recreational matches against each other, a spectacle few of the inhabitants of the garden worlds want to miss out on. Even the royal governing the planet sometimes engages in these festivals, although winning against them is seen as an action against the empire and is punishable by death.\\
\\
There exists also a type of noble called the Rune-Caster, scientists with an immense understanding of Meta-Space, who use special patterns and motives breach into Meta-Space and evoke powerful \say{spells}. While most Rune-Casters are seen tending to the various Meta-Space related mechanisms and systems used in the Terrastrocraty, a few take active rolls on the battlefield, casting devastation onto their foes, blocking their Meta-Spatial communications, or sending troops through Meta-Space behind enemy-lines.

\pagebreak


\subsection{The Amazing\textsuperscript{TM} Corporation - I am here to interest you in our Products\textsuperscript{TM}.}

\subsection{The Liberation Fleet - I am ascended through merit.}
Surveying the cosmos, this enormous, fully self-sustaining fleet travels from star to star, recruiting new members into their ranks, spreading their ideas of liberation. They see themselves as bringers of peace and prosperity to all who join them, and take in everyone with open arms. They are lead by the Consuls, a council of governors and planners, making up roughly 2\% of the fleets population. It is their mission to expand and perpetuate the fleet, and coordinate its one singular effort: ascension. They have a powerful understanding of Meta-Space and intertwine this knowledge with their own beliefs of an omniscient creator entity, which they dubbed Primus, that once resided in the higher dimensions. They further postulate, that this entity shattered itself under the strain of creating the universe, which explains the absence of its direct influence from the universe, and why it is, as they say, imperfect. These splinters of Primus, according to them, also are what lead to the emergence of sentience across the universe, as they believe that these divine fragments merge themselves into the basic minds of sufficiently developed, and therefore complex creatures across the universe, bestowing them with self-awareness in the process. They think that this was a way for Primus to save himself from total annihilation. Now they see it as their sacred duty to reform Primus, and try to achieve this by ascending, as they put it, into Meta-Space, and merging their souls together, until the being of Primus re-emerges, and is strong enough to complete the process, by forcefully extracting what belongs to him out of every single being in the universe, and then to create a new, perfect one, armed with the knowledge of his first failed attempt.

\subsection{The  Veritas Communes - I'm looking for truth and inspiration. }

\subsection{\textit{I} - \textit{I} Am Creations Archivist.}

\subsection{The Hominid Alliance - A long lost legacy}

\subsection{??? - I do not yet know.}

\chapter{Warriors of the Future}

\chapter{Scenarios}

\chapter{How To participate}


\end{document}